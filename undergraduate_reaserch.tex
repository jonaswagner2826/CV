\section{Undergraduate Research Experience}

\cventry{Fall 2018 - Spring 2020}
{Projects Involving Machine Learning and Virtual Reality}
{Electrical and Computer Engineering}
{University of Wisconsin-Platteville}
{Advisor: Dr. Mehdi Roopaei}
{
\textbf{Disaster Response Applications (ML, Edge Analytics, and VR)}
\begin{itemize}
    \item Wrote several grant proposals (approx. \$15 K awarded) that funded  research into the use of ML and edge analytics within a multi-agent framework for disaster response
    \item Developed a virtual framework to develop and test an object detection algorithm
    \item Working on training a neural network using the Darknet framework to perform object detection on a custom database
    \item Submitted a manuscript detailing this virtual framework to the IEEE 10\textsuperscript{th} Annual Computing and Communication Workshop and Conference
\end{itemize}
\textbf{Computer Vision at the Edge on a Jetson Nano}
\begin{itemize}
    \item Explored the Jetson Nano Platform and worked within a Linux environment
    \item Used existing tools to connect a CSI camera and detect faces using Haar classifiers
\end{itemize}
\textbf{Applying VR to Education}
\begin{itemize}
    \item Assisted in the preliminary development of a VR framework for distance education
    \item Assisted other students in creating a dynamic system visualization platform to provide students with an interactive environment to understand dynamic system modeling
\end{itemize}
\textbf{Exploring Unity ML Agents}
\begin{itemize}
    \item Worked with Unity ML Agents to learn about ML and reinforcement learning methods
    \item Used pre-trained ML models and explored how well agents could perform the same objective in modified virtual environments
\end{itemize}
}

\cventry{Spring 2019 - Spring 2020}
{Implementing K-Means and EM-Algorithm in MATLAB and Python}
{Electrical and Computer Engineering}
{University of Wisconsin-Platteville}
{Advisor: Dr. Hynek Boril}
{Learned about fundamental statistical modeling and ML techniques while also learning Python
\begin{itemize}
    \item Implemented K-means Clustering and the EM-Algorithm to statistically model data
    \item Used Windows Subsystem for Linux and Midnight Commander to run Python naively
\end{itemize}
}

\cventry{Fall 2018 - Summer 2020}
{Computational Analysis of MEMS Pressure Sensors}
{Engineering Physics}
{University of Wisconsin-Platteville}
{Advisor: Dr. Gokul Gopalakrishnan}
{Evaluated the limitations of different methods used for modeling the behavior of silicon nanomembranes for MEMS pressure sensing applications
\begin{itemize}
    \item Focused primarily on automating the computation and analysis process
    \item Used ANSYS Workbench to perform FEM analysis on single crystalline silicon membranes under uniform pressure
    \item Used Python (NumPy and pandas) to automate data analysis
    \item Created plots to visualize data with matplotlib
\end{itemize}
}

\cventry{June 2019}
{LabVIEW Programming of a Mobile Robot}
{Mechanical Engineering}
{University of the West of Scotland - Paisley}
{Advisor: Dr. Luc Rolland}
{Short-term study abroad research trip: Worked on developing a control algorithm for a sbRIO controlled robot that avoids obstacles and maps an environment autonomously}